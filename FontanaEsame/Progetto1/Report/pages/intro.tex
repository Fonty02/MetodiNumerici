\section{Introduzione}
\subsection{Traccia Progetto 1}
\begin{itemize}
    \item 

Leggere le seguenti due immagini in Matlab, usando i comandi:

\begin{verbatim}
    imt1 = imread('Bern1.bmp');
    imt2 = imread('Bern2.bmp');
    
    if(size(imt1,3)>1)
        imt1 = rgb2gray(imt1);
    end
    if(size(imt2,3)>1)
        imt2 = rgb2gray(imt2);
    end
\end{verbatim}

 \item Costruire la matrice della differenza, avendo prima trasformato i dati in tipo double:
 \begin{verbatim}
    imt1 = double(imt1)
    imt2 = double(imt2)
    Diff_Mat = abs(imt2-imt1)    
 \end{verbatim}

 
 \item Calcolare la SVD troncata di Diff
 Mat, in particolare trascrivere in Matlab almeno tre criteri
 per selezionare automaticamente il numero di componenti k da trattenere.
 \item Calcolare l’errore relativo in norma 2 e in norma di Frobenius per l’approssimazione ottenuta.
 \item Trasformare la matrice dell’approssimazione in un vettore colonna e scalare i valori tra 0 e 1.
 \item Applicare un algoritmo di binarizzazione variando la soglia (per un esempio, si veda il file
 Grad img.m).
 \item Visualizzare l’immagine finale, fancendo un opportuno “reshape” e utilizzando il comando imshow.
 L’immagine cos`ı ottenuta dovrebbe avere in bianco i pixel che identificano zone di cambiamento
 tra l’immagine I1 e l’immagine I2, mentre in nero, i pixel relativi alle zone non cambiate.
 \item Commentare i risultati

\end{itemize}

\subsection{Dati iniziali}
La matrice \textbf{Diff\_Mat} è una matrice di dimensione 301x301, in quanto le immagini \textbf{Bern1.bmp} e \textbf{Bern2.bmp} sono immagini di tale dimensione.\\
Il rank della matrice \textbf{Diff\_Mat} è 301 dunque massimo. Il che significa che tutti i valori singolari sono non nulli.\\
Le matrici U, S* e V hanno dimensioni rispettivamente 301x301, 301x301 e 301x301.\\

*La matrice S è una matrice diagonale contenente i valori singolari, pertanto è possibile usare una memorizzazione per matrici diagonali.\\